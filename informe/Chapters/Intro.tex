% Chapter Template

\chapter{Introducci\'on} % Main chapter title

\label{Chapter1}

%----------------------------------------------------------------------------------------
%   SECTION 1
%----------------------------------------------------------------------------------------

\section{An\'alisis de los datos}

EL trabajo consiste en desarrollar un algoritmo que permita predecir el puntaje (1-5) que un usuario
otorg\'o a un producto en base a su review (t\'itulo, texto, etc).

Para ello, vamos a hacer enfoque en el campo del ``sentiment analysis'', que consiste en poder
interpretar las emociones plasmadas en el texto.

Comenzamos utilizando algoritmos simples que nos permitieran hacer un an\'alisis r\'apido de los
datos, con el fin de poder estimar que tipos de algoritmos funcionan mejor, como preprocesar el
texto de forma que los resultados mejoren, etc.

%-----------------------------------
%   SUBSECTION 1
%-----------------------------------
\subsection{Preprocesamiento del set de entrenamiento}

Comenzamos por divir el set de entrenamiento en 3 partes:\\


\begin{itemize}
\setlength\itemsep{0em}
  \item train: consiste en un 80\% del set original. Es utilizado para entrenar los algoritmos.
  \item cv: consiste en un 10\% del set original. Es utilizado para ajustar los hiper par\'ametros de los
        algoritmos y reducir el overfitting al set de entrenamiento.
  \item test: consiste en el 10\% restante. Utilizado evaluar el desempe\~no final del algoritmo.
\end{itemize}

De esta forma podemos tener una estimaci\'on confiable sobre los resultados obtenidos sin necesidad
de hacer varios submits a Kaggle.

%-----------------------------------
%   SUBSECTION 2
%-----------------------------------

\subsection{Preprocesamiento de los campos}

Cada review contiene una serie de campos con informaci\'on sobre el mismo:\\

\begin{itemize}
\setlength\itemsep{0em}
  \item Id - El id que identifica a cada review
  \item ProductId - El Id del producto
  \item UserId - El Id del usuario
  \item ProfileName - El nombre del usuario
  \item HelpfulnessNumerator - El numerador indicando la cantidad de usuarios que juzgaron al review como util
  \item HelpfulnessDenominator - El denominador indicando la cantidad de usuarios que evaluaron si el review fue útil o no
  \item Prediction - La cantidad de estrellas del review
  \item Time - Un timestamp para el review
  \item Summary - Un resumen del review
  \item Text - Texto del review
\end{itemize}

Encontramos que exceptuando Text y Summary, poco aportan los otros campos (y algunas veces hasta
entorpecen el aprendizaje).

%----------------------------------------------------------------------------------------
%   SECTION 2
%----------------------------------------------------------------------------------------

\section{Sentiment analysis}

Los puntajes otorgados a cada review se basan en que tan satisfechos estuvieron los usuarios con
el producto, por lo que los comentarios deber\'ian expresar cu\'al fue el sentimiento (alegr\'ia,
enojo, frustraci\'on, etc) al recibirlo.

Es por esto que consideramos que la finalidad de TP es basicamente hacer un sentiment analysis, y
luego mapear los sentimientos al valor del review.

En el sentiment analysis el preprocesamiento del texto juega un rol my importante para la gran
mayor\'ia de los algoritmos \cite{importance_of_preprocessing}.

El primer paso es la tokenizaci\'on, que consiste en separar el texto en palabras u otros s\'imbolos
que pudan aparecer (URLs, emoticones, puntos, comas, tags HTML, etc) \cite{mining_twitter_data}.

Los emoticones pueden jugar un rol importante, por lo que, para evitar perderlos y poder
interpretarlos mejor, se puden reemplazar por alguna palabra, por ejemplo:

\begin{itemize}
\setlength\itemsep{0em}
  \item :), :-) -> smile
  \item :( -> sad
  \item etc
\end{itemize}

Un paso bastante com\'un, es remover las llamadas ``stop words'', que son palabras muy comunes en
los textos, y por lo general no aportan mucho al significado general de la oraci\'on \cite{stopwords}.
