% Chapter Template

\chapter{Desarrollo} % Main chapter title

\label{Desarrollo} % Change X to a consecutive number; for referencing this chapter elsewhere, use \ref{ChapterX}

%----------------------------------------------------------------------------------------
%   SECTION 1
%----------------------------------------------------------------------------------------

\section{Algoritmos usados}

Para comenzar, se decidió hacer uso de algoritmos simples y ver el resultado obtenido para poder
compararlos y decidir cuál de todos profundizar.

Las pruebas se realizaron con los subsets obtenidos de dividir el set original, como se explicó en
la introducción.

%-----------------------------------
%   SUBSECTION 1
%-----------------------------------
\subsection{KNN}

Uno de los primeros algoritmos a evaluar fue KNN. Para calcular las distancias, se decidió encodear
los textos en un DTM utilizando una biblioteca de R \cite{dtm_for_sentiment_analysis}.

Previa a la creación de la matriz, se hizo una limpieza del texto, removiendo stop-words, puntuaciones
HTML, etc.

Se comenzó probando con una porción muy pqueña del set (5\%), y se incrementó el tamaño en las pruebas
subsiguientes.

Al llegar a aproximadamente un 15\% del set, el algoritmo era demasiado lento para clasificar (algunas
horas), y no daba buenos resultados, por lo que se decidió probar otra alternativa.

Probablemente se pueda implementar mas adelante utilizando doc2vec, lo que debería reducir las
dimensiones de los textos, y además dar resultados más precisos debido a la natureleza de los
vectores generados (más similares al ser más parecidos).

%-----------------------------------
%   SUBSECTION 2
%-----------------------------------

\subsection{Random forest}

Este algritmo se comportó bastante mejor que KNN, ya que su tiempo de entrenamiento/clasificación
era muy inferior, pero los resultados no fueron demasido buenos para predecir calificaciones menores
a 3. 

Si bien la idea fue tomada de \cite{random_forest_movie_reviews}, se convirtieron los textos a un
bag of words y luego se la utilizó para entrenar el forest.

Se puede analizar más adelante aplicar una técnica similar a la del artículo citado.
